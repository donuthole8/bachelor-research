%%%%!TEX root = ../Thesis.tex
%!TEX program = xelatex
\documentclass[../Thesis]{subfiles}

% 本文

\begin{document}
\chapter{まとめ}
\section{結論}
	本研究では、災害時における被害把握にてドローンの有効性を示すため、ドローン空撮映像を用いた災害領域検出手法を提案し、ドローン空撮画像のみから斜面崩壊・浸水領域検出を行った。補助データを用いることなく災害を検出することで、先行研究で問題になっていた山間部における補助データの位置合わせのずれによる精度低下を解決した。また、指標を用いた閾値処理によって建物領域を除去することで、先行研究では想定していなかった山間部での浸水検出に成功した。
	\quad 提案手法ではF値にて70\%以上を達成でき、ほぼ全ての精度指標にて先行研究と同等、または同等以上の精度が得られたため、災害時の新たな観測手段としてドローンの利用が期待できる結果となった。

\section{今後の課題}
\subsection{斜面崩壊・浸水領域検出}
	災害領域検出にて、斜面崩壊と浸水領域同士の誤検出が発生するという問題がある。本研究では画像特徴による各指標を用いて領域分類を行った。輝度が高く彩度が低い領域は浸水領域であるが、太陽光の当たり方や光量の影響により浸水領域を斜面崩壊領域として判別してしまうことがある。
	
	⇒なんか、あれしたらいいと思う地図で斜面を斜面崩壊、平地は浸水、河川は河川、すごいイト思う	
	
	しかし、輝度や彩度等は天候条件や太陽光の当たり方によって値が変化することがあり誤検出を発生させる原因となる。

\subsection{浸水領域分類}
	浸水領域に関し,既存の河川と災害によって水没した浸水箇所の判別は未実装である.
	
	
	豪雨災害後の河川と浸水した土地の画像特徴は非常に似ており、画像特徴や指標を用いた判別は困難である。よって、地図等の補助データを用いて既存の河川領域を判別する手法を検討する。ただし、先行研究にて問題となっている	位置合わせの際のずれによる精度低下が生じないように位置合わせを行う手法を検討する必要がある。
	
	
	
	地図データを用いで河川領域を除去することが挙げられる
	浸水領域検出にて、既存の河川と浸水した土地の領域の判別が未実装である。


	本実験ではドローン空撮映像から斜面崩壊・浸水領域を検出する手法を提案した.現状の問題点として斜面崩壊・浸水領域同士の誤検出が発生する点が挙げられる.また,浸水領域に関し,既存の河川と災害によって水没した浸水箇所の判別は未実装である.これらの問題への対処として,新たな指標によって領域の判別精度を高めることや,位置合わせの際にずれが生じないように地図データを用いで河川領域を除去することが挙げられる.今後はこれらの誤検出改善・実装を進め,システムの精度向上を目指す.

\subsection{地図への反映}
	ドローンやヘリコプター空撮映像を用いた災害領域検出を行う最終目的は災害救助支援であり、その一環として被災把握地図の作成が存在する。

\subsection{研究テーマの拡張【ドローン空撮映像を用いた被災領域把握地図作成の自動化システム】}
	最終的にはコレ
	これできたらまあ意義あるんじゃないかなあああ

	ヘリの代替手段じゃなくて、ドローンにしかできない詳細把握、被害地図作成、院では即時性を押さず、ドローンの役割に徹する(衛星⇒ヘリ⇒ドローン)

\subsection{土砂とか畑とかどうするか}

\subsection{瓦礫・建物領域除去の精度}

\subsection{その他人工物の誤検出}

\subsection{閾値自動決定}



\end{document}
