%%%%!TEX root = ../Thesis.tex
%!TEX program = xelatex
\documentclass[../Thesis]{subfiles}

% 本文

\begin{document}
\chapter{まとめ}
\section{結論}
	本研究では,災害時における被害把握にてドローンの有効性を示すため,ドローン空撮映像を用いた災害領域検出手法を提案し,ドローン空撮画像のみから斜面崩壊・浸水領域検出を行った.補助データを用いることなく災害を検出することで,先行研究で問題になっていた補助データと空撮画像の位置合わせのずれによる精度低下を解決した.また,指標を用いた閾値処理による建物領域の除去を行ったが,建物の検出に失敗する例があり,改善の余地がある結果となった.提案手法ではほぼすべての精度評価値にて70\%以上を達成し,F値にて先行研究と同等以上の精度が得られたため,災害時の観測手段としてドローンの利用が期待できる結果となった.

\section{今後の課題}
\subsection{災害領域検出について}
	本研究では画像特徴による各指標を用いて領域分類を行った.しかし,災害領域検出にて,斜面崩壊と浸水領域同士の誤検出が発生するという問題がある.これは3.5.1項で述べたように画像中の光量や影の偏りによるものだと思われる.よって,輝度の補正処理の検討が考えられる.2.3節で述べたヒストグラム均一化は複数枚の画像を用いて本手法を適用した場合に,同一の閾値を用いて処理を行うことを可能にするために画素値の分布を補正する処理であった.これに対し,単一の画像内の全体の輝度を均一化することで,画像全体の光量や影による明るさの差異を除去する処理による対処が考えられる.
	
% 	⇒なんか,あれしたらいいと思う地図で斜面を斜面崩壊,平地は浸水,河川は河川,すごいイト思う	
\subsection{河川領域除去について}
	浸水領域に関し,既存の河川と災害によって水没した浸水箇所の判別は未実装である.豪雨災害後の河川と浸水した土地の画像特徴は類似していることが多く,特徴量や指標を用いた判別は困難である.よって,地図のような補助データを用いて既存の河川領域を判別する手法を検討する.ただし,先行研究にて問題となっている,補助データと空撮画像の位置合わせのずれによる精度低下が少ないような位置合わせ手法を検討する必要がある.そのため,斜め視点の画像を直下視点に補正することで,位置合わせのずれを無くす手法などが考えられる.

\subsection{土砂領域除去について}
	斜面崩壊領域に関し,既存の土砂と災害によって土砂が崩壊した箇所の判別は未実装である.前節と同様に豪雨災害後の土砂と斜面崩壊箇所の画像特徴は類似しており,本手法での判別は難しい.よって,災害前の直下視点空撮画像を補助データとして利用し,災害前にも土砂である箇所を非災害領域として判別する手法を検討する.ただし,前節と同様に位置合わせのずれが生じないように位置合わせを行う必要がある.
	
% 	DEMだと斜面の土砂は無理だし,平地になだれ込んだ斜面崩壊を検知できない

\subsection{建物領域検出について}
    建物領域検出に関し,本研究ではGSIを用いた閾値処理によって建物領域を除去した.しかし,実験2において建物領域を除去することに失敗した.これは3.5.2項で述べたようにGSI指標がドローン空撮画像に適さなかったためであると考える.よって,別の指標や画像特徴,補助データを用いる手法を検討する.まず,円形度とエッジ抽出率を利用する手法が挙げられる.円形度は形状が複雑であるほど値が小さくなるため,建物の屋根や壁といった単純な形状領域では値が大きくなる.エッジ抽出率は均一度が高いほど値が大きくなるため建物の判別に有効であるが,空や水面なども表面が均一であるため円形度のような他指標と併用する必要がある.次に,建物データを用いる手法が挙げられる.前節と同様に位置合わせのずれが生じないように建物データと空撮画像の位置合わせを行い,建物領域を検出する手法が有効であると考えられる.

\subsection{災害領域の地図への反映}
	ドローンやヘリコプター空撮映像を用いて災害領域検出を行う最終目的はは救助活動や二次災害防止のための被災状況の把握であり,その一環として被災状況分布図の作成がある.これを作成することで救助や二次災害防止などに活用ができる.しかし,現在ではドローンによって災害領域を撮影した後,被災状況分布図を作成する作業が手動で行われている\cite{art02}.本研究では単一の斜め視点のフレーム画像から災害領域を検出したが,オルソ補正を行った直下視点画像に対し本手法を適用し,地図に災害領域を重ね合わせることで被災状況分布図作成の自動化手法を検討する.

% \subsection{研究テーマの拡張【ドローン空撮映像を用いた被災領域把握地図作成の自動化システム】}
% 	最終的にはコレ
% 	これできたらまあ意義あるんじゃないかなあああ

% 	ヘリの代替手段じゃなくて,ドローンにしかできない詳細把握,被害地図作成,院では即時性を押さず,ドローンの役割に徹する(衛星⇒ヘリ⇒ドローン)

\end{document}
