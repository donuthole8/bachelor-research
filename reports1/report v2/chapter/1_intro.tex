%!TEX root = ../Thesis.tex
%!TEX program = xelatex
\documentclass[../Thesis]{subfiles}

% 本文
\begin{document}
\chapter{はじめに}
\label{cha:はじめに}
\section{背景}
  近年,日本では豪雨による斜面崩壊・浸水被害が多発しており,斜面崩壊・浸水被害が問題となっている.気象庁の統計では,1時間降水量50mm以上の大雨の平均年間発生回数は,統計初期の10年間(1976~1985年)に比べ最近10年間(2011~2020年)で約1.4倍に増加した\cite{web01}.また,平成29年7月九州北部豪雨では大雨による斜面崩壊や浸水により1761棟の住宅が被害を受けており(平成30年2月21日時点)\cite{art01},被害箇所を早急に把握することは救助活動の効率化や二次災害防止などに有効であるとされている.この被災状況の把握に関し,安全な位置からの解析が可能なリモートセンシング技術が注目されている\cite{art00}.\\
  \quad リモートセンシング技術による災害領域検出には主に人工衛星,有人航空機(以降,ヘリコプター),無人航空機(以降,ドローン)が用いられる.人工衛星は広範囲の把握が可能であり,画像処理において扱いが容易な直下視点の画像が入手可能である.しかし,解像度が低いため詳細な情報の入手が困難であり,天候や撮影周期によっては画像が得られないという問題がある.ヘリコプターは人工衛星に比べ災害発生直後に画像を取得でき,解像度においても優れている.しかし,金銭的コストが非常に高く,周囲に発着場が必要であるという問題がある.また,保有台数が少なく災害箇所を網羅できない可能性があり,悪天候時には出動できないこともある.これに対しドローンは安価かつ迅速に解像度の高い画像の取得が可能であるため,被害箇所の早急な把握に有効である.しかし,現状の活用事例ではドローンの操縦から取得したデータの解析までの作業が全て手動で行われており,運用にかかる労力が問題となっている\cite{art02}. \\
%   \quad よって,本研究では災害時の新たな観測手段としてドローンを活用し,災害領域を検出することを考える.

  % また,2020年9月8日時点で43都道府県の消防本部がドローンを導入しており災害時の利用が期待されている.
  % 最初に使える画像の制限を明記(特定の条件やこの研究が意味を成す条件,この研究が適用できる条件等)


\section{先行研究}
  リモートセンシング技術による斜面崩壊・浸水領域検出に関する研究を以下に示す.

\subsection{ヘリコプター空撮画像を用いた斜面崩壊領域検出}
  中山ら\cite{art03}は地震災害後のヘリコプター空撮画像を用いて斜面崩壊領域を検出する手法を提案している.L*a*b*色空間にて土砂領域を検出し,テクスチャ特徴の一つである異質度と傾斜情報をDEMデータから取得し道路や平地の誤検出を低減している.この手法では解像度の高さを利用し,異質度にて人工物を除去することができるという利点がある.しかし,位置情報を含み,直下視点である衛星画像と比べ,DEMデータとの位置合わせのずれにより精度が低下するという問題がある.

\subsection{ヘリコプター空撮画像を用いた浸水領域検出}
  雨宮ら\cite{art04}は豪雨災害後のヘリコプター空撮画像を用いて浸水領域を検出する手法を提案している.この手法ではエッジ抽出率と異質度が低い領域を浸水領域として検出する.建物データを用いて限定された建物候補領域から,形状が単純であるほど大きい値を示す円形度にて建物領域を除去することができるという利点がある.しかし,この手法は都市部での浸水領域検出を想定しており,山間部では建物データの位置合わせが難しいため建物除去の精度が低下する.また,位置合わせを手動で行うため労力がかかるという問題がある.


\section{本研究の目的} 
  以上の背景と先行研究を踏まえ,災害時の観測手段としてドローンの活用における有効性を示す.本研究では,災害後の高解像度ドローン空撮画像のみから斜面崩壊・浸水領域を自動で検出する手法を提案する.補助データを用いず,画像特徴のみを利用することで先行研究で問題になっていたDEMデータと空撮画像の位置合わせのずれによる精度低下を抑え,手動にて位置合わせを行う労力の低減が可能である. また,ドローンの有効性を示すため,先行研究と同等の精度で災害領域を検出することを目的とする.
  
  %   \quad よって,本研究では災害時の新たな観測手段としてドローンを活用し,災害領域を検出することを考える.
  
  % 本研究では,災害後の高解像度ドローン空撮画像のみから斜面崩壊・浸水領域を自動検出する手法を提案する.補助データを用いず, 画像特徴のみを利用することで


\section{本論文の構成}
  \label{sec:本論文の構成}
  本論文の構成を以下に示す. \par
  第1章では本研究の背景,先行研究,及び目的について述べた.\par
  第2章では本研究の提案手法について述べる.\par
  第3章では実験方法及び実験結果について述べる.\par
  第4章ではまとめとして結論及び今後の課題について述べる.

\end{document}
